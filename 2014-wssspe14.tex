%% Editorial Revision Requests:
%% A deeper exploration of the Khmer lifecycle, as noted by the reviewer, would
%% strengthen the paper

%% Additional major revisions as noted by the reviewer
%% should be considered optional, but the paper's discussion of the khmer
%% lifecycle and its ``growing pains'' would better adjust it to fit with the
%% collection

%% Minor revisions:
%% Some reordering of discussion
%% Extension of the abstract
%% Additional citations as noted in reviewer comments

%% 'detailed and deeper information about the tool’s growth such as users, 
%% communities, usage statistics, effort areas would be interesting and 
%% informative'

% 'the language of the paper feels too informal and should be improved.
% Colloquialisms should be avoided such as: ``bioinformaticians invent a
% new format every 5 minutes on average''


\documentclass[12pt]{article}
\usepackage{simplemargins}
\usepackage{times}
\usepackage{graphicx}
\usepackage[small,compact]{titlesec}
\usepackage{hyperref}
\usepackage{fancyhdr}
\usepackage{paralist}
\fancypagestyle{firststyle}{%
  \fancyhf{}% Clear header/footer
  \fancyhead[L]{Experience Report}
}
\pagestyle{empty}

\date{April 2014}


\setlength{\parindent}{0pt}
\setlength{\parskip}{0.70ex}
\setallmargins{1in}
\title{Channeling community contributions to scientific software: a hackathon experience}

\author{Michael R. Crusoe$^{1}$\\
C. Titus Brown$^{1,2\ast}$\\
\small \bf{1} Microbiology and Molecular Genetics,\\
\small \bf{2} Computer Science and Engineering,\\
\small Michigan State University, East Lansing, MI, USA\\
\small $\ast$ Corresponding author: ctb@msu.edu}

\begin{document}
\maketitle
\thispagestyle{firststyle}

\abstract{The khmer software project provides both research and
  production functionality for large-scale nucleic-acid sequence
  analysis.  In 2014, we participated in a two-day global sprint
  coordinated by the Mozilla Science Labs, and offered a mentored
  experience in contributing to a scientific software project.  We
  provided entry-level tasks and worked with contributors as they
  claimed issues, created new branches, committed changes, issued pull
  requests, went through code review, and iterated until their
  contributions were merged.  In this experience paper, we describe
  the process, relate anecdotal experiences, and suggest frameworks
  for other projects that want to attract entry-level contributors.
  The khmer software is developed openly at {\sf
    http://github.com/ged-lab/khmer/}.}

\setlength{\parindent}{0pt}
\setlength{\parindent}{0pt}
\setlength{\parskip}{0.70ex}

\section{Introduction}

We have been granted a week-long extension so that we can do the
hackathon on July 22nd and July 23rd --CTB and MRC.

\end{document}
