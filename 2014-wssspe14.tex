\documentclass[11pt]{article}
\usepackage{simplemargins}
\usepackage{times}
\usepackage{graphicx}
\usepackage[small,compact]{titlesec}
\usepackage{hyperref}
\usepackage{fancyhdr}
\usepackage{paralist}
\fancypagestyle{firststyle}{%
  \fancyhf{}% Clear header/footer
  \fancyhead[L]{Experience Report}
}
\pagestyle{empty}

\date{August 2014}


\setlength{\parindent}{0pt}
\setlength{\parskip}{0.70ex}
\setallmargins{1in}
\title{Channeling community contributions to scientific software: a sprint experience}

\author{Michael R. Crusoe$^{1}$\\
C. Titus Brown$^{1,2\ast}$\\
\small \bf{1} Microbiology and Molecular Genetics,\\
\small \bf{2} Computer Science and Engineering,\\
\small Michigan State University, East Lansing, MI, USA\\
\small $\ast$ Corresponding author: ctb@msu.edu}

\begin{document}
\maketitle
\thispagestyle{firststyle}

\abstract{In 2014, the khmer software project participated in a
  two-day global sprint coordinated by the Mozilla Science Lab. We
  offered a mentored experience in contributing to a scientific
  software project for anyone who was interested.  We provided
  entry-level tasks and worked with contributors as they worked
  through our development process.  The experience was successful on
  both a social and a technical level, bringing in 13 contributions
  from 9 new contributors and validating our development process.  In
  this experience paper we describe the sprint preparation and
  process, relate anecdotal experiences, and draw conclusions about
  what other projects could do to enable a similar outcome.  The khmer
  software is developed openly at {\sf
    http://github.com/ged-lab/khmer/}.}

\setlength{\parindent}{0pt}
\setlength{\parindent}{0pt}
\setlength{\parskip}{0.70ex}

\section{Introduction}

% @CTB maybe change title.
Sustainable development of scientific software inevitably depends on
following good software development practices.  However, even
rudimentary development practices such as version control and testing
are rarely a formal part of scientific training.  One way to learn
these practices is to participate in an open source project, which
often provide a path for new contributors to get involved.  Open
source {\em scientific} software projects can go further by providing
scientists the opportunity to work on a science-focused project.

In July 2014, Mozilla Science Lab ran a two-day global ``sprint'' for
a wide variety of software projects. As part of this sprint the khmer
project offered a mentored software contribution experience.  The
khmer project is a bioinformatics library developed primarily at
Michigan State University, and it uses many open source software
development practices \cite{khmer,2013-wssspe13}.  These practices
include open development and code review on GitHub using a workflow
called GitHub Flow \cite{githubflow}, the maintenance of a large suite
of unit and functional tests, continuous integration, formal release
testing, and semantic versioning.  The two authors of this paper, MRC
and CTB, are respectively the lead software engineer and the principle
investigator on the NIH grant that funds MRC and khmer development.

We had several concerns when organizing our part of the sprint.  We
were uncertain how to target the list of issues for an unknown number
of developers with a potentially wide range of development experience.
We were also concerned that our development process involved too many
steps for new developers to work through.  Finally, we were unsure of
whether this would be an effective use of our time.  Despite these
reservations, we participated in the sprint because the sprint would
be an opportunity to use new developers to expose problems in our
documentation and software.  We also took advantage of the sprint to
ask local lab members to go through the full development cycle
themselves.

The global sprint was organized as follows: each physical location was
asked to provide directions for attendees, along with coffee, Internet
connections, and a video wall.  Mozilla then connected these into a
global video wall, and also provided a central IRC channel for the
sprint.  The goal of the sprint was to provide a supportive
environment in which to collaborate on open science projects,
encourage contributions from new people, and introduce new people to a
variety of projects.  Standard work hours, breaks for lunch and
snacks, and an emphasis on acknowledging a diversity of contributions
were all part of the setup.  There was also a pan-site Code of Conduct
provided, which was widely advertised and may have led to more
significant buy-in from certain communities.

In the end, the two-day sprint was a modest success technically, and a
big success socially. We merged 13 contributions from 9 distinct
contributors into the master development branch; we solved a
previously unappreciated installation problem with our software; and
we revamped our development documentation to include a detailed guide
to getting started.  Overall, we felt that the sprint was a very
useful investment of our time and energy and are looking forward to
future sprints.

Below, we describe the pre-sprint preparation, the sprint itself, and the
post-sprint outcomes.  We then provide some concluding thoughts.

\section{Pre-sprint preparation}

We announced the sprint in a blog post \cite{blogpost}, broadcast the
blog post via Twitter, evangelized it at two conferences, and entered
it into the Mozilla Science Lab project list.  We then provided an
issue on our GitHub issue tracker for people to subscribe to for
updates.  This provided a more specific notification channel than a
mailing list for us to use in informing contributors of our plans, and
also ensured that interested parties already had a GitHub account.

We next designated a set of issues with a ``low-hanging fruit'' tag.
These issues were chosen (or in some cases designed) to be
entry-level: they required no biology or bioinformatics knowledge, and
no prior experience with the khmer project was needed.  The issues
targeted a range of Python, C++, and documentation changes.  For
example, one issue involved replacing a C++ stdlib exception with a
khmer specific one, while another issue required copying an existing
test and making a minor modification.

Finally, we wrote a detailed walk-through for new
contributors.\footnote{http://khmer.readthedocs.org/en/docs-hackathon/dev/getting-started.html}
This walk-through assumed some prior command-line expertise and basic
familiarity with git, but otherwise required no particular familiarity
with GitHub, the GitHub Flow process, khmer installation, or anything
specific to khmer development.  Crucially, much of this workflow was
written to be copy-paste at the command line, which avoided the
burdensome requirement for inexperienced developers to compose many
new commands.

The workflow covered twenty five distinct steps and included forking a
copy of khmer on GitHub, cloning it locally, building khmer, running
the tests, claiming an issue, making changes and committing them,
verifying the changes by running the tests again, pushing back to
GitHub, and going through continuous integration and code review.

\section{During the sprint}

The sprint ran for two days, July 22 and 23, from 9am to 5pm EST
(Michigan local time).  We merged 13 pull requests that were both
started and finished during this period, contributed by five remote
and four local users.  Our in-person contributors included someone
from industry who took vacation days to attend the sprint; several
people loosely affiliated with the lab but who had not previously
contributed to the codebase; and another member of the MSU community
who was unaffiliated with our lab.

\paragraph{Activities}
During the sprint, we interacted with participants, revised the
central sprint issue, and updated our documentation regularly in
response to problems.  CTB primarily focused on updating the
documentation while MRC primarily interacted with remote and local
developers.

We enforced a requirement that each contributor completed all the
items on our development checklist, just like any other contributor.
However, MRC found that it was difficult to balance detailed code
review with the many different demands on his time as multiple
contributors updated issues, encountered problems, and had questions.
This was an area where more code reviewers would be needed to scale.

\paragraph{Communication and Environment} Throughout the sprint, Internet Relay Chat (IRC) provided a real-time
venue for private and group chat that supported our issue-driven
process.  We had little direct interaction over video, because we were
a small part of the larger Mozilla Science Lab sprint.  However, the
sense of community and cooperation was greatly enhanced by the
presence of the always-on video wall.

The environment was friendly and relaxed, with a welcoming physical
environment and good community feeling.  We shared the sprint space
with a Data Carpentry sprint as well, which helped build community feeling.
We used Twitter to announce first-time contributors on the first day, and
CTB provided a running commentary as issues came up and were addressed.

\paragraph{Issues and problems}
The most important issue that surfaced during the sprint was that our
test running commands simply didn't work for many.  We had included
some installation commands in the 'make test' command that depended on
certain versions of Python build infrastructure.  Many of our sprint
participants ran into this problem in the first two hours of the
sprint, leading us to debug and change the installation commands live,
and then update the instructions.

We also found that many contributors did not reliably follow the
detailed instructions.  This was not surprising -- 25 steps is quite a
lot! -- but we couldn't find a way to simplify our workflow, either.
However, because we were providing support in real time, we could
almost always give useful feedback to help participants discover which
steps they had missed and correct them.  A longer feedback cycle might
have led to many orphaned pull requests as contributors gave up on our
workflow.

\section{Post-sprint feedback and actions}

We had 9 participants who both started and finished a total of 13 pull
requests during the sprint; five were remote, and four were
local.

None of our nine participants had experienced GitHub Flow before, and
most had no prior exposure to testing or code review.  Several
participants expressed enthusiasm for having gone through the process.
A further 3 participants are still working on finishing pull requests
started during the sprint.

We are further revising the documentation after post-sprint review, to
better link different sections and refine sections that were updated
hastily during the sprint.

\section{Discussion and concluding thoughts}

We felt that the most valuable part of the sprint was in setting aside
this focused time for in-lab problem solving and collaboration.  Most of
the khmer developers were in the room together and when a problem
needed to be discussed (e.g. the installation problem) it was easy to
hold an impromptu meeting.  This is different from our usual lab
development process which is largely asynchronous.

The rapid, systematic review, improvement and testing of documentation
was tremendously valuable; having put 10 or more participants through
our ``getting started'' documentation means that we are now certain
that the instructions work!  However, more observation of
inexperienced contributors will undoubtedly lead to areas where can
optimize the documentation for first-time participants.

In the long term, we do not expect many, or perhaps any, of the sprint
participants to continue developing on khmer.  None of the
participants external to the lab worked in our subfield of biology,
and khmer itself has a fairly narrow set of functionality.  However,
we can guess that because of the improved documentation, khmer will
now be better able to attract contributions from developers who {\em
  are} interested in longer-term engagement with the project.

We do hope that the sprint participants will use their new experience
with GitHub, distributed version control, and remote development to
contribute to other open source projects.  We plan to query their
GitHub activity on public projects to see if there is additional
engagement in the months to come.

The presence of existing process and infrastructure let us work with
new contributors more easily than we would have been able to a year
ago: they got more done.  In turns this meant that we could leverage
their contributions more easily: we gained more from what they did.
Process documentation, issue tracking, tests, reliable build and test
instructions, and mechanisms for support were all important.  The
up-front organization specific to our sprint was minimal, because we
already had many existing resources.  Moreover, the getting-started
guide and the low-hanging-fruit issues provide an excellent entree
into our software project that remains after the sprint.

It was important to have two active, dedicated participants so that
specific issues (pull requests and technical support) as well as
meta-issues (documentation and communication) could be handled.  We
believe the process would not have scaled much beyond 2-3 simultaneous
participants without an additional khmer developer, which could be a
bottleneck for projects; perhaps our next khmer sprint will focus on
training new code reviewers!

The biggest unresolved challenge is how to more effectively walk
participants through their first contribution.  While 25 steps may
seem overly complex, each step is an important part of the development
cycle; experienced software developers may elide many of the steps
mentally (``of {\em course} I run the tests after each commit!'') but
they are all necessary.  This complexity illuminates the challenge
facing scientists who want to learn basic software development
practices: each development practice (e.g. using version control, or
testing, or code review) requires that many different steps be
executed in combination.  Our experience from the sprint suggests that
participants can be taught to execute these steps fairly easily, if
sufficient time and support is provided.

Future revisions to our on-boarding documentation could simplify the
documentation in a few ways by eliminating optional steps (e.g. our
current documentation provides instructions for using ccache).  Apart
from that, we could more formally study the ``first-time contributor''
workflow by working with people as they go through it, to see where
mistakes are commonly made.  We are wary of oversimplifying, however,
because simplifying further could result in increased maintenance
burden on our part, and also diminish the ability of people to
transition from our project (which uses a fairly standard GitHub-flow
based workflow) to others.

We are looking forward to future sprints and would like to involve more
scientific software development groups in teaching others about their
development workflows.

% @figure <= project perspective 

% @define Github
% @define git

\section*{Acknowledgments}

MRC has been funded by AFRI Competitive Grant no. 2010-65205-20361
from the USDA NIFA and is now funded by the National Human Genome
Research Institute of the National Institutes of Health under Award
Number R01HG007513; both to C. Titus Brown.

\bibliographystyle{plain}
\bibliography{2014-wssspe14}

\end{document}
